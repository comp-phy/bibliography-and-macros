%\makeatletter
% \@ifpackageloaded{tensor}% tensor is a package for a better typesetting of tensors
% {
% \renewcommand{\tnsr@Aux}[3][]{%
% \mathpalette{\tnsr@Plt{#1}{#3}}{\mathrm #2}%
% \tnsr@Wrn
% }%\tnsr@Aux
% }{%
% \relax%
% }
% \makeatother


% operators
\DeclareMathOperator{\divergence}{div}
\DeclareMathOperator{\Divergence}{Div}
\DeclareMathOperator{\gradient}{grad}
\DeclareMathOperator{\Gradient}{Grad}
\DeclareMathOperator{\rot}{rot}
\DeclareMathOperator{\Asym}{Asym}
\DeclareMathOperator{\Sym}{Sym}
\DeclareMathOperator{\Tr}{Tr}
\DeclareMathOperator{\signum}{sign}
\DeclareMathOperator{\supp}{supp}
\DeclareMathOperator{\cof}{cof}

% jump
\newcommand{\jumpdis}[1]{\ensuremath{\left\lsem #1 \right\rsem}} % difference between function values at the point of jump discontinuity

% hyperbolic functions
\DeclareMathOperator{\arcsinh}{arcsinh}
\DeclareMathOperator{\arccosh}{arccosh}
\DeclareMathOperator{\arctanh}{arctanh}
\DeclareMathOperator{\arccoth}{arccoth}

% invariants of second order tensor
\DeclareMathOperator{\invariantI}{I_1}
\DeclareMathOperator{\invariantII}{I_2}
\DeclareMathOperator{\invariantIII}{I_3}

% big o
\newcommand{\bigo}[1]{\ensuremath{O\left(#1 \right)}}
\newcommand{\smallo}[1]{\ensuremath{o\left(#1 \right)}}

% exponential
\newcommand{\exponential}[1]{\ensuremath{{\mathrm e}^{#1}}}

% imaginary unit
\newcommand{\iunit}{\ensuremath{\mathrm{i}}}


% real and imaginary part
\newcommand{\realp}{\mathrm{real}}
\newcommand{\imagp}{\mathrm{imag}}

%\newcommand{\Real}{\Re}
%\newcommand{\Imag}{\Im}
\providecommand{\Real}{\Re}
\providecommand{\Imag}{\Im}

% predicates
\newcommand{\charac}{\ensuremath{\mathrm{char}}} % characteristic quantity such as length scale, etc.
\newcommand{\reference}{\mathrm{ref}}
\newcommand{\crit}{\mathrm{crit}}
\newcommand{\bydefinition}{\mathrm{def}}
\newcommand{\traceless}[1]{{#1}_{\delta}}

% dimensionless variables and functions
\newcommand{\dimless}[1]{#1^\star}

% derivatives
\newcommand{\diff}{\mathrm{d}}
\newcommand{\Diff}[1]{\mathrm{D}_{#1}} % For Frechet and Gateaux derivative

% inexact differential
\newcommand{\dbar}{{\mathchar'26\mkern-12mu \diff}}
\newcommand{\idiff}{\dbar}

% body
\newcommand{\body}{{\mathcal B}}

% vectors and tensors
\renewcommand{\vec}[1]{\ensuremath{\mathbf{#1}}}
\newcommand{\greekvec}[1]{\ensuremath{\boldsymbol{#1}}}
\makeatletter
\@ifpackageloaded{bm}% 
{\renewcommand{\vec}[1]{\ensuremath{\bm{#1}}}%
\renewcommand{\greekvec}[1]{\ensuremath{\bm{#1}}}%
}{%
\relax% do nothing
}
\makeatother
\newcommand{\tensorq}[1]{\ensuremath{\mathbb{#1}}}      % tensorial quantity
\newcommand{\tensorc}[1]{\ensuremath{\mathrm{#1}}}      % tensorial quantity components  

\newcommand{\conjugate}[1]{#1^\star}
\newcommand{\transpose}[1]{#1^\top}
\newcommand{\transposei}[1]{#1^{-\top}}
\newcommand{\inverse}[1]{#1^{-1}}

% Identity matrix
\newcommand{\identity}{\ensuremath{\tensorq{I}}}

% Cauchy stress
\newcommand{\cstress}{\tensorq{T}}
\newcommand{\cstressc}{\tensorc{T}}

\newcommand{\ecstress}{\tensorq{S}}
\newcommand{\ecstressc}{\tensorc{S}}

% First Piola stress tensor
\newcommand{\fpstress}{\tensorq{T}_{\mathrm{R}}}
\newcommand{\fpstressc}{\tensorc{T}_{\mathrm R}}

% Second Piola--Kirchhoff stress tensor
\newcommand{\spstress}{\tensorq{S}_{\mathrm{R}}}
\newcommand{\spstressc}{\ensuremath{{{\mathrm S}_{\mathrm R}}}}

% Couple stress tensor
\newcommand{\couplestress}{\tensorq{M}}
\newcommand{\couplestressc}{\tensorc{M}}

% deformation, deformation gradient
\newcommand{\deformation}{\greekvec{\chi}}
\newcommand{\deformationc}{\tensorc{\chi}}

\newcommand{\fgrad}{\tensorq{F}}
\newcommand{\fgradc}{\tensorc{F}}
\newcommand{\fgradrel}[3][]{\fgrad^{#1}_{#2}\left(#3\right)}

% displacement
\newcommand{\displacement}{\vec{U}}
\newcommand{\displacementc}{\tensorc{U}}

% right Cauchy-Green tensor
\newcommand{\rcg}{\tensorq{C}}
\newcommand{\rcgc}{\tensorc{C}}        
\newcommand{\rcgrel}[3][]{\rcg^{#1}_{#2}\left(#3\right)}

% left Cauchy-Green tensor
\newcommand{\lcg}{\tensorq{B}}
\newcommand{\lcgc}{\tensorc{B}}        
\newcommand{\lcgrel}[3][]{\lcg^{#1}_{#2}\left(#3\right)}

%\newcommand{\piolastrain}{\tensorq{b}} % Piola deformation tensor (inverse of right Cauchy--Green)
%\newcommand{\fingerstrain}{\tensorq{c}} % Finger deformation tensor (inverse of left Cauchy--Green)

% rotation
\newcommand{\rotation}{\tensorq{R}}
\newcommand{\rotationrel}[3][]{\rotation^{#1}_{#2}\left(#3\right)}

% stretch
\newcommand{\stretchu}{\tensorq{U}}
\newcommand{\stretchurel}[3][]{\stretchu^{#1}_{#2}\left(#3\right)}
\newcommand{\stretchv}{\tensorq{V}}
\newcommand{\stretchvrel}[3][]{\stretchv^{#1}_{#2}\left(#3\right)}

% linearized strain (symmetric part of displacement gradient), skew-symmetric part of displacement gradient
\makeatletter
\@ifpackageloaded{bm}% 
{%
\newcommand{\linstrain}{\bbespilon} %requires \usepackage[bbgreekl]{mathbbol}
% YES, the spelling is wrong, but this is how it is coded in the package
}{%
\newcommand{\linstrain}{\tensorq{\varepsilon}}
}

\@ifpackageloaded{bm}%
{%
\newcommand{\skewdgradient}{\bbomega} 
}{%
\newcommand{\skewdgradient}{\tensorq{\omega}}
}

\@ifpackageloaded{bm}%
{%
\newcommand{\linstress}{\bbtau} % stress in linearise elasticity
}{%
\newcommand{\linstress}{\tensorq{\tau}}
}
\makeatother

\newcommand{\linstrainc}{\mathrm{\varepsilon}}
\newcommand{\linstressc}{\mathrm{\tau}}
\newcommand{\skewdgradientc}{\mathrm{\omega}}

% Lagrangean and Eulerian strain
\newcommand{\lstrain}{\tensorq{E}} % Green--Saint-Venant strain
\newcommand{\lstrainc}{\tensorc{E}} % Green--Saint-Venant strain, components
\newcommand{\estrain}{\tensorq{e}} % Euler--Almansi strain, components
\newcommand{\estrainc}{\tensorc{e}} % Euler--Almansi strain, components

% Rivlin-Ericksen tensor
\newcommand{\rivlin}{{\tensorq{A}}}

% generic tensor quantity
\newcommand{\generictensor}{{\tensorq{A}}}
\newcommand{\generictensorc}{\tensorc{A}} % component of the tensor

% deviatoric part of Cauchy stress
\newcommand{\dcstress}{\cstress - \left( \frac{1}{3}\Tr \cstress \right) \identity}
\newcommand{\dcstresssymb}{\traceless{\cstress}}

% mean normal stress
\newcommand{\cstressnorm}{\frac{1}{3}\Tr \cstress}

% velocity and velocity gradient, (skew)symmetric part of velocity gradient
\newcommand{\vecv}{\ensuremath{\vec{v}}}
\newcommand{\gradv}{\ensuremath{\nabla \vecv}}
\newcommand{\gradasym}{\ensuremath{\tensorq{W}}}
\newcommand{\gradsym}{\ensuremath{\tensorq{D}}}
\newcommand{\dgradsymsymb}{\ensuremath{\gradsym_{\delta}}}
\newcommand{\gradvl}{\ensuremath{\tensorq{L}}}

% surface velocity
\newcommand{\unders}[1]{\ensuremath{\underaccent{\mathrm{s}}{#1}}}

\newcommand{\gradsymop}{\nabla_{\mathrm{sym}}}
\newcommand{\gradasymop}{\nabla_{\mathrm{asym}}}

\newcommand{\vecvc}{\tensorc{v}}

% velocity and velocity gradient, (skew)symmetric part of velocity gradient, COMPONENTS
\newcommand{\gradsymc}{\tensorc{D}}
\newcommand{\gradasymc}{\tensorc{W}}

% functionals
\newcommand{\functional}[1]{{\mathfrak #1}}
\newcommand{\history}[3]{{\functional{#1}_{#2}^{#3}}}

% base vectors
\newcommand{\bvec}[1]{\vec{e}_{#1}} % current configuration
\newcommand{\Bvec}[1]{\vec{E}_{#1}} % reference configuration

% dual base vectors
\newcommand{\bvecd}[1]{\vec{e}^{#1}} % current configuration
\newcommand{\Bvecd}[1]{\vec{E}^{#1}} % reference configuration

% Cartesian basis, current configuration
\newcommand{\bvecx}{\bvec{\hat{x}}}
\newcommand{\bvecy}{\bvec{\hat{y}}}
\newcommand{\bvecz}{\bvec{\hat{z}}}

% Cartesian basis, reference configuration
\newcommand{\BvecX}{\Bvec{\hat{X}}}
\newcommand{\BvecY}{\Bvec{\hat{Y}}}
\newcommand{\BvecZ}{\Bvec{\hat{Z}}}

% Cartesian dual basis, reference configuration
\newcommand{\BvecdX}{\Bvecd{\hat{X}}}
\newcommand{\BvecdY}{\Bvecd{\hat{Y}}}
\newcommand{\BvecdZ}{\Bvecd{\hat{Z}}}

% Cartesian dual basis, current configuration
\newcommand{\bvecdx}{\bvecd{\hat{x}}}
\newcommand{\bvecdy}{\bvecd{\hat{y}}}
\newcommand{\bvecdz}{\bvecd{\hat{z}}}

% same as above but now in cylindrical coordinates
\newcommand{\bvecr}{\bvec{\hat{r}}}
\newcommand{\bvect}{\bvec{\hat{\theta}}}
\newcommand{\bvecp}{\bvec{\hat{\varphi}}}
%\newcommand{\bvecz}{\bvec{\hat{z}}}

\newcommand{\bvecdr}{\bvecd{\hat{r}}}
\newcommand{\bvecdt}{\bvecd{\hat{\theta}}}
\newcommand{\bvecdp}{\bvecd{\hat{\varphi}}}

\newcommand{\BvecR}{\Bvec{\hat{R}}}
\newcommand{\BvecP}{\Bvec{\hat{\Phi}}}
%\newcommand{\BvecZ}{\Bvec{\hat{Z}}}

\newcommand{\BvecdR}{\Bvecd{\hat{R}}}
\newcommand{\BvecdP}{\Bvecd{\hat{\Phi}}}
%\newcommand{\BvecdZ}{\Bvecd{\hat{Z}}}

% components
\newcommand{\vhatx}[1][\vecvc]{{#1}^{\hat{x}}}
\newcommand{\vhaty}[1][\vecvc]{{#1}^{\hat{y}}}
%\newcommand{\bvhatz}{\vhat{e}_{\hat{z}}}

\newcommand{\vhatr}[1][\vecvc]{{#1}^{\hat{r}}}
\newcommand{\vhatt}[1][\vecvc]{{#1}^{\hat{\theta}}}
\newcommand{\vhatp}[1][\vecvc]{{#1}^{\hat{\varphi}}}
\newcommand{\vhatz}[1][\vecvc]{{#1}^{\hat{z}}}

% indices
\newcommand{\hatx}{\hat{x}}
\newcommand{\haty}{\hat{y}}
\newcommand{\hatz}{\hat{z}}
\newcommand{\hatr}{\hat{r}}
\newcommand{\hatp}{\hat{\varphi}}
\newcommand{\hatt}{\hat{\theta}}
\newcommand{\hatX}{\hat{X}}
\newcommand{\hatY}{\hat{Y}}
\newcommand{\hatZ}{\hat{Z}}

% inner and outer radius (for some calcualtions)
\newcommand{\Rin}{R_{\mathrm{in}}}
\newcommand{\Rout}{R_{\mathrm{out}}}
\newcommand{\rin}{r_{\mathrm{in}}}
\newcommand{\rout}{r_{\mathrm{out}}}
 
% base vectors, abstract covariant and contravariant basis
\newcommand{\cobvec}[1]{\vec{g}_{#1}} % covariant base vector
\newcommand{\conbvec}[1]{\vec{g}^{#1}} % contravariant base vector
\newcommand{\cobvecn}[1]{\vec{g}_{\hat{#1}}} % covariant base vector
\newcommand{\conbvecn}[1]{\vec{g}^{\hat{#1}}} % contravariant base vector

\newcommand{\mtensor}{\tensorq{g}}  % metric tensor
\newcommand{\mtensorc}{{\mathrm g}} % metric tensor, components

% Christoffel symbols
\newcommand{\christoffel}[2]{\tensor{\Gamma}{^{#1}_{#2}}}

% mean curvature
\newcommand{\meancurvature}{\mathrm{K}} % mean curvature

\newcommand{\mtensorref}{\tensorq{G}}  %metric tensor in reference configuration
\newcommand{\mtensorrefc}{{\mathrm G}} %metric tensor in reference configuration, components

% Kronecker delta, Levi--Civitta symbol
\newcommand{\kdelta}[1]{\tensor{\delta}{#1}}
\newcommand{\lcepsilon}[1]{\tensor{\epsilon}{#1}}

% distributions
\newcommand{\diracdelta}{\delta}
\newcommand{\Heaviside}{H}

% hypergeometric function
\newcommand{\hypergeom}[4]{\ensuremath{ \mathrm{F}\left( \left[#1, #2 \right]; \left[ #3 \right]; #4\right)}}

% sets
\newcommand{\R}{\ensuremath{{\mathbb R}}}
\makeatletter
%\@ifpackageloaded{hyperref}% \C is defined in hyperref package
%{\renewcommand{\C}{\ensuremath{{\mathbb C}}}%
%}{%
%\newcommand{\C}{\ensuremath{{\mathbb C}}}%
%}
\makeatother
%\renewcommand{\C}{\ensuremath{{\mathbb C}}}% The lines above are no longer needed?
\newcommand{\Q}{\ensuremath{{\mathbb Q}}}
\newcommand{\N}{\ensuremath{{\mathbb N}}}
\newcommand{\Z}{\ensuremath{{\mathbb Z}}}


% Reynolds, Womersley number, etc.
\newcommand{\Reynolds}{\mathrm{Re}}
\newcommand{\Womersley}{\mathrm{Wo}}
\newcommand{\Rayleigh}{\mathrm{Ra}}
\newcommand{\RayleighSqrt}{\mathrm{R}}
\newcommand{\Prandtl}{\mathrm{Pr}}
\newcommand{\Grashof}{\mathrm{Gr}}
\newcommand{\Mach}{\mathrm{Ma}}
\newcommand{\Froude}{\mathrm{Fr}}
\newcommand{\Peclet}{\mathrm{Pe}}
\newcommand{\Eckert}{\mathrm{Ec}}
\newcommand{\Brinkman}{\mathrm{Br}}
\newcommand{\Nusselt}{\mathrm{Nu}}

% Young modulus, Poisson ratio
\newcommand{\Young}{\mathrm{E}}
\newcommand{\Poisson}{\mathrm{\nu}}

% Symetric and antisymetric tensors
\newcommand{\asym}[1]{\ensuremath{\Asym \left( #1 \right)}}
\newcommand{\sym}[1]{\ensuremath{\Sym \left( #1 \right)}}

% Energy, free energy, entropy, temperature
\newcommand{\tenergy}{\ensuremath{e}_{\mathrm{tot}}} % specific total energy (energy per unit mass), sum of specific internal energy and the specific kinetic energy
\newcommand{\ienergy}{\ensuremath{e}} % specific internal energy (energy per unit mass)
\newcommand{\menergy}{\ensuremath{e}_{\mathrm{mech}}} % specific mechanical energy (energy per unit mass), kinetic energy plus internal energy minus thermal contribution
\newcommand{\kenergy}{\ensuremath{e_{\mathrm{kin}}}} % specific kinetic energy (kinetic energy per unit mass)
\newcommand{\fenergy}{\ensuremath{\psi}} % specific free energy
\newcommand{\entropy}{\ensuremath{\eta}} % specific entropy
\newcommand{\entalphy}{\ensuremath{h}} % specific enthalpy
\newcommand{\gibbs}{\ensuremath{g}} % specific Gibbs free energy

\newcommand{\temp}{\ensuremath{\theta}} % temperature
\newcommand{\thpressure}{\ensuremath{p_{\mathrm{th}}}} % thermodynamic pressure
\newcommand{\mns}{\ensuremath{m}} % mean normal stress
\newcommand{\temptoref}{\ensuremath{\vartheta}} % (temperature - reference temperature)/(reference temperature)

% Net energy, free energy, entropy, ...
\newcommand{\nettenergy}{\ensuremath{E}_{\mathrm{tot}}} % net total energy
\newcommand{\netmenergy}{\ensuremath{E}_{\mathrm{mech}}} % net mechanical energy
\newcommand{\netthenergy}{\ensuremath{E}_{\mathrm{therm}}} % net thermal energy
\newcommand{\netienergy}{\ensuremath{E}} % net internal energy
\newcommand{\netkenergy}{\ensuremath{E_{\mathrm{kin}}}} % net kinetic energy
\newcommand{\netentropy}{\ensuremath{S}} % net entropy
\newcommand{\netheat}{\ensuremath{Q}} % net heat

% Specific molar gas constant
\newcommand{\Rspecific}{\ensuremath{R_{\mathrm{s}}}}
\newcommand{\Rmol}{\ensuremath{R_{\mathrm{m}}}}
 
% Specific heat at constant volume 
\newcommand{\cheatvol}{\ensuremath{c_{\mathrm{V}}}}

% Density in reference configuration
\newcommand{\rhor}{\rho_{\mathrm{R}}}

% Energy flux, heat flux, entropy flux
\newcommand{\efluxc}{\vec{j}_{e}} % energy flux, current configuration
\newcommand{\eflux}{\vec{J}_{e}} % energy flux, reference configuration

\newcommand{\hfluxc}{\vec{j}_{q}}     % heat flux, current configuration
\newcommand{\hfluxcc}{\tensorc{j}_{q}}     % heat flux, current configuration, components
\newcommand{\hflux}{\vec{J}_{q}}     % heat flux, reference configuration

\newcommand{\entfluxc}{\vec{j}_{\entropy}} % entropy flux, current configurtion 
\newcommand{\entflux}{\vec{J}_{\entropy}} % entropy flux, reference configuration

% Energy source, entropy source
\newcommand{\esourcec}{\ensuremath{q_{e}}} % energy source, current configuration
\newcommand{\hsourcec}{\ensuremath{q}} % heat source, current configuration
\newcommand{\entsourcec}{\ensuremath{q_{\entropy}}} % entropy source, current configuration

% Thermodynamical fluxes and affinities
\newcommand{\thfluxc}[1]{\vec{j}_{#1}} % thermodynamic flux, current configuration
\newcommand{\thaffinityc}[1]{\vec{a}_{#1}} % thermodynamic affinity, current configuration

% Entropy production
\newcommand{\entprodc}{\xi} % entorpy production, current configuration
\newcommand{\entprodctemp}{\zeta} % entorpy production times temperature, current configuration

% Derivatives, partial derivatives, covariant derivatives 
\newcommand{\pd}[2]{\ensuremath{\frac{\partial {#1}}{\partial {#2}}}}
\newcommand{\ppd}[2]{\ensuremath{\frac{\partial^2 {#1}}{\partial {#2^2}}}}
\newcommand{\dd}[2]{\ensuremath{\frac{\diff {#1}}{\diff {#2}}}}
\newcommand{\cd}[2]{\ensuremath{\frac{\diff^* {#1}}{\diff {#2}}}}
\newcommand{\ddd}[2]{\ensuremath{\frac{\diff^2 {#1}}{\diff {#2}^2}}}

% Upper convected (Oldroyd) derivative
\newcommand{\fid}[1]{\ensuremath{\accentset{\triangledown}{#1}}}
\newcommand{\fidd}[1]{\ensuremath{\accentset{\triangledown \! \triangledown}{#1}}}

% Lower convected derivative
\newcommand{\lfid}[1]{\ensuremath{\accentset{\meddiamond}{#1}}}
\newcommand{\lfidd}[1]{\ensuremath{\accentset{\meddiamond \! \meddiamond}{#1}}}

% Jaumann derivative
\newcommand{\jfid}[1]{\ensuremath{\accentset{\vartriangle}{#1}}}
\newcommand{\jfidd}[1]{\ensuremath{\accentset{\vartriangle \! \vartriangle}{#1}}}

% Green--Naghdi derivative
\newcommand{\gfid}[1]{\ensuremath{\accentset{\medsquare}{#1}}}
\newcommand{\gfidd}[1]{\ensuremath{\accentset{\medsquare \! \medsquare}{#1}}}

% Truesdell derivative
\newcommand{\tfid}[1]{\ensuremath{\accentset{\medcircle}{#1}}}
\newcommand{\tfidd}[1]{\ensuremath{\accentset{\medcircle \! \medcircle}{#1}}}

% Material derivative (\dot with \overline)
\newcommand{\mdif}[1]{\ensuremath{\dot{\overline{#1}}}}

\makeatletter
\@ifpackageloaded{tensor}% tensor is a package for a better typesetting of tensors
{
\newcommand{\codev}[2]{\ensuremath{\left. {#1} \right|\indices{_{#2}}}}
}{%
\newcommand{\codev}[2]{\ensuremath{\left. {#1} \right|_{#2}}}
}
\makeatother

\makeatletter
\@ifpackageloaded{tensor}% tensor is a package for a better typesetting of tensors
{
\newcommand{\contradev}[2]{\ensuremath{\left. {#1} \right|\indices{^{#2}}}}
}{%
\newcommand{\contradev}[2]{\ensuremath{\left. {#1} \right|^{#2}}}
}
\makeatother


% Bessel and Kelvin functions

\newcommand{\BesselI}[2]{\ensuremath{{\mathrm I}_{#1}\left(#2\right)}} 
\newcommand{\BesselK}[2]{\ensuremath{{\mathrm K}_{#1}\left(#2\right)}}
\newcommand{\BesselJ}[2]{\ensuremath{{\mathrm J}_{#1}\left(#2\right)}}
\newcommand{\BesselY}[2]{\ensuremath{{\mathrm Y}_{#1}\left(#2\right)}}

\newcommand\BesselRoot[2]{\ensuremath{{\rm j}_{#1,#2}}}

\newcommand{\KelvinBer}[2]{\ensuremath{{\mathrm{ber}}_{#1}\left(#2\right)}} 
\newcommand{\KelvinBei}[2]{\ensuremath{{\mathrm{bei}}_{#1}\left(#2\right)}}
\newcommand{\KelvinKer}[2]{\ensuremath{{\mathrm{ker}}_{#1}\left(#2\right)}}
\newcommand{\KelvinKei}[2]{\ensuremath{{\mathrm{kei}}_{#1}\left(#2\right)}}

% Chebyshev polynominals
\newcommand{\Chebyshevp}[3]{\ensuremath{{\mathrm T}_{#1}^{#2}\left(#3\right)}} 
\newcommand{\Chebyshev}[2]{\Chebyshevp{#1}{}{#2}} 

% norms
\newcommand{\norm}[2][]{\ensuremath{\left\|#2\right\|_{#1}}}
\newcommand{\absnorm}[1]{\ensuremath{\left|#1\right|}}

% volume
\makeatletter
\@ifundefined{volume}{%
\newcommand{\volume}[1][\Omega]{\ensuremath{#1}}}%
{%
\renewcommand{\volume}[1][\Omega]{\ensuremath{#1}}}
\makeatother

% surface and volume elements (reference configuration)
\newcommand{\svolume}[1][\Omega]{\ensuremath{\partial #1}}
\newcommand{\volumee}{\diff \mathrm{V}}
\newcommand{\surfacee}{\diff \vec{S}}
\newcommand{\surfacees}{\diff \mathrm{S}}
\newcommand{\linee}{\diff \vec{X}}

% surface and volume elements (current configuration)
\newcommand{\cvolumee}{\diff \mathrm{v}}
\newcommand{\csurfacee}{\diff \vec{s}}
\newcommand{\csurfacees}{\diff \mathrm{s}}
\newcommand{\clinee}{\diff \vec{x}}

% volume and surface integral
\newcommand{\intvolume}[2][\volume]{\int_{#1} #2\; \volumee}
\newcommand{\intsvolume}[2][\svolume]{\int_{#1} #2\; \surfacee}

% surface Jacobian
\newcommand{\surfacej}{\mathrm{j}}

% products
\newcommand{\tensortensor}[2]{\ensuremath{#1 \otimes #2}}
\makeatletter
\@ifpackageloaded{MnSymbol} % : as binary operator needs MnSymbol package
{
\newcommand{\tensordot}[2]{\ensuremath{#1 \vdotdot #2}} 
}{%
\newcommand{\tensordot}[2]{\ensuremath{#1 : #2}} 
}
\makeatother
\newcommand{\vectordot}[2]{\ensuremath{#1 \bullet #2}}
\newcommand{\vectorcross}[2]{\ensuremath{#1 \times #2}}

% function spaces
\newcommand{\scont}[2][\Omega]{\ensuremath{{\mathcal C}^{#2} \left(#1 \right)}}
\newcommand{\sdist}[1][\Omega]{\ensuremath{{\mathcal D} \left(#1 \right)}}
\newcommand{\sdistd}[1][\Omega]{\ensuremath{{\mathcal D}^\prime \left(#1 \right)}}

\newcommand{\scdiv}[1][\Omega]{\ensuremath{{\mathcal V} \left(#1 \right)}}

\newcommand{\loc}{\mathrm{loc}}

\newcommand{\slebl}[2]{\ensuremath{L}^{#1}_{\loc} \left(#2 \right)}
\newcommand{\sleb}[2]{\ensuremath{L}^{#1} \left(#2 \right)}


\newcommand{\ssob}[3]{\ensuremath{W}^{#1, #2} \left(#3 \right)}

% dualities, scalar products
\newcommand{\fadual}[4]{\left\langle #1, #2\right\rangle_{#3, #4}}
\newcommand{\fascal}[4]{\left\langle #1, #2\right\rangle_{#3, #4}}

% dual space
\newcommand{\dspace}[1]{#1^{\star}}

% tensorial function
\newcommand{\tensorf}[1]{{\mathfrak{#1}}}

% normal stress differences
\newcommand{\firstnsd}{N_1}
\newcommand{\secondnsd}{N_2}
