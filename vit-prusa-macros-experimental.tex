%\makeatletter
% \@ifpackageloaded{tensor}% tensor is a package for a better typesetting of tensors
% {
% \renewcommand{\tnsr@Aux}[3][]{%
% \mathpalette{\tnsr@Plt{#1}{#3}}{\mathrm #2}%
% \tnsr@Wrn
% }%\tnsr@Aux
% }{%
% \relax%
% }
% \makeatother


% operators
\DeclareMathOperator{\divergence}{div}
\DeclareMathOperator{\Divergence}{Div}
\DeclareMathOperator{\rot}{rot}
\DeclareMathOperator{\Asym}{Asym}
\DeclareMathOperator{\Sym}{Sym}
\DeclareMathOperator{\Tr}{Tr}
\DeclareMathOperator{\signum}{sign}
\DeclareMathOperator{\supp}{supp}


% hyperbolic functions
\DeclareMathOperator{\arcsinh}{arcsinh}
\DeclareMathOperator{\arccosh}{arccosh}
\DeclareMathOperator{\arctanh}{arctanh}
\DeclareMathOperator{\arccoth}{arccoth}

% invariants of second order tensor
\DeclareMathOperator{\invariantI}{I_1}
\DeclareMathOperator{\invariantII}{I_2}
\DeclareMathOperator{\invariantIII}{I_3}

% moments of second order tensor
\DeclareMathOperator{\invariantIm}{\tilde{I}_1}
\DeclareMathOperator{\invariantIIm}{\tilde{I}_2}
\DeclareMathOperator{\invariantIIIm}{\tilde{I}_3}

% big o
\newcommand{\bigo}[1]{\ensuremath{O\left(#1 \right)}}
\newcommand{\smallo}[1]{\ensuremath{o\left(#1 \right)}}

% exponential
\newcommand{\exponential}[1]{\ensuremath{{\mathrm e}^{#1}}}

% imaginary unit
\newcommand{\iunit}{\ensuremath{\mathrm{i}}}


% real and imaginary part
\newcommand{\realp}{\mathrm{real}}
\newcommand{\imagp}{\mathrm{imag}}

%\newcommand{\Real}{\Re}
%\newcommand{\Imag}{\Im}
\providecommand{\Real}{\Re}
\providecommand{\Imag}{\Im}

% predicates
\newcommand{\reference}{\mathrm{ref}}
\newcommand{\crit}{\mathrm{crit}}
\newcommand{\bydefinition}{\mathrm{def}}
\newcommand{\traceless}[1]{#1_{\delta}}

% derivatives
\newcommand{\diff}{\mathrm{d}}

% body
\newcommand{\body}{{\mathcal B}}

% vectors and tensors
\renewcommand{\vec}[1]{\ensuremath{\mathbf{#1}}}
\newcommand{\greekvec}[1]{\ensuremath{\boldsymbol{#1}}}
\makeatletter
\@ifpackageloaded{bm}% 
{\renewcommand{\vec}[1]{\ensuremath{\bm{#1}}}%
\renewcommand{\greekvec}[1]{\ensuremath{\bm{#1}}}%
}{%
\relax% do nothing
}
\makeatother
\newcommand{\tensorq}[1]{\ensuremath{\mathbb{#1}}}      % tensorial quantity
\newcommand{\tensorc}[1]{\ensuremath{\mathrm{#1}}}      % tensorial quantity components  

\newcommand{\transpose}[1]{#1^\top}
\newcommand{\transposei}[1]{#1^{-\top}}
\newcommand{\inverse}[1]{#1^{-1}}

% Identity matrix
\newcommand{\identity}{\ensuremath{\tensorq{I}}}

% Cauchy stress
\newcommand{\cstress}{\tensorq{T}}
\newcommand{\cstressc}{\ensuremath{\mathrm T}}

\newcommand{\ecstress}{\tensorq{S}}
\newcommand{\ecstressc}{\ensuremath{\mathrm S}}

% First Piola stress tensor
\newcommand{\fpstress}{\tensorq{T}_{\mathrm{R}}}
\newcommand{\fpstressc}{\ensuremath{{{\mathrm T}_{\mathrm R}}}}

% deformation, deformation gradient
\newcommand{\deformation}{\greekvec{\chi}}
\newcommand{\fgrad}{\tensorq{F}}
\newcommand{\fgradrel}[3][]{\fgrad^{#1}_{#2}\left(#3\right)}

% right Cauchy-Green tensor
\newcommand{\rcg}{\tensorq{C}}
\newcommand{\rcgrel}[3][]{\rcg^{#1}_{#2}\left(#3\right)}

% left Cauchy-Green tensor
\newcommand{\lcg}{\tensorq{B}}
\newcommand{\lcgrel}[3][]{\lcg^{#1}_{#2}\left(#3\right)}

% rotation
\newcommand{\rotation}{\tensorq{R}}
\newcommand{\rotationrel}[3][]{\rotation^{#1}_{#2}\left(#3\right)}

% stretch
\newcommand{\stretchu}{\tensorq{U}}
\newcommand{\stretchurel}[3][]{\stretchu^{#1}_{#2}\left(#3\right)}
\newcommand{\stretchv}{\tensorq{V}}
\newcommand{\stretchvrel}[3][]{\stretchv^{#1}_{#2}\left(#3\right)}

% linearized strain
\newcommand{\linstrain}{\tensorq{\varepsilon}}

% Lagrangean and Eulerian strain
\newcommand{\lstrain}{\tensorq{E}}
\newcommand{\estrain}{\tensorq{e}}

% Rivlin-Ericksen tensor
\newcommand{\rivlin}{{\tensorq{A}}}

% generic tensor quantity
\newcommand{\generictensor}{{\tensorq{A}}}

% deviatoric part of Cauchy stress
\newcommand{\dcstress}{\cstress - \left( \frac{1}{3}\Tr \cstress \right) \identity}
\newcommand{\dcstresssymb}{\traceless{\cstress}}

% mean normal stress
\newcommand{\cstressnorm}{\frac{1}{3}\Tr \cstress}

% velocity and velocity gradient, (skew)symmetric part of velocity gradient
\newcommand{\vecv}{\ensuremath{\vec{v}}}
\newcommand{\gradv}{\ensuremath{\nabla \vecv}}
\newcommand{\gradasym}{\ensuremath{\tensorq{W}}}
\newcommand{\gradsym}{\ensuremath{\tensorq{D}}}
\newcommand{\dgradsymsymb}{\ensuremath{\gradsym_{\delta}}}
\newcommand{\gradvl}{\ensuremath{\tensorq{L}}}

\newcommand{\gradsymop}{\nabla_{\mathrm{sym}}}
\newcommand{\gradasymop}{\nabla_{\mathrm{asym}}}

\newcommand{\vecvc}{\tensorc{v}}

% velocity and velocity gradient, (skew)symmetric part of velocity gradient, COMPONENTS
\newcommand{\gradsymc}{\ensuremath{\mathrm D}}

% functionals
\newcommand{\functional}[1]{{\mathfrak #1}}
\newcommand{\history}[3]{{\functional{#1}_{#2}^{#3}}}

% base vectors
\newcommand{\bvec}[1]{\vec{e}_{#1}}

\newcommand{\bvecx}{\bvec{\hat{x}}}
\newcommand{\bvecy}{\bvec{\hat{y}}}
%\newcommand{\bvecz}{\bvec{\hat{z}}}

\newcommand{\bvecr}{\bvec{\hat{r}}}
\newcommand{\bvect}{\bvec{\hat{\theta}}}
\newcommand{\bvecp}{\bvec{\hat{\varphi}}}
\newcommand{\bvecz}{\bvec{\hat{z}}}

% components
\newcommand{\vhatx}[1][v]{{#1}^{\hat{x}}}
\newcommand{\vhaty}[1][v]{{#1}^{\hat{y}}}
%\newcommand{\bvhatz}{\vhat{e}_{\hat{z}}}

\newcommand{\vhatr}[1][v]{{#1}^{\hat{r}}}
\newcommand{\vhatt}[1][v]{{#1}^{\hat{\theta}}}
\newcommand{\vhatp}[1][v]{{#1}^{\hat{\varphi}}}
\newcommand{\vhatz}[1][v]{{#1}^{\hat{z}}}

% indices
\newcommand{\hatx}{\hat{x}}
\newcommand{\haty}{\hat{y}}
\newcommand{\hatz}{\hat{z}}
\newcommand{\hatr}{\hat{r}}
\newcommand{\hatp}{\hat{\varphi}}
\newcommand{\hatt}{\hat{\theta}}
 
% base vectors, abstract covariant and contravariant basis
\newcommand{\cobvec}[1]{\vec{g}_{#1}} % covariant base vector
\newcommand{\conbvec}[1]{\vec{g}^{#1}} % contravariant base vector
\newcommand{\cobvecn}[1]{\vec{g}_{\hat{#1}}} % covariant base vector
\newcommand{\conbvecn}[1]{\vec{g}^{\hat{#1}}} % contravariant base vector

\newcommand{\mtensor}{\tensorq{g}}
\newcommand{\mtensorc}{{\mathrm g}}

% hypergeometric function
\newcommand{\hypergeom}[4]{\ensuremath{ \mathrm{F}\left( \left[#1, #2 \right]; \left[ #3 \right]; #4\right)}}

% sets
\newcommand{\R}{\ensuremath{{\mathbb R}}}
% \makeatletter
% \@ifpackageloaded{hyperref}% \C is defined in hyperref package
% {\renewcommand{\C}{\ensuremath{{\mathbb C}}}%
% }{%
% \newcommand{\C}{\ensuremath{{\mathbb C}}}%
% }
% \makeatother
% %\newcommand{\C}{\ensuremath{{\mathbb C}}}% The lines above are no longer needed?

\makeatletter
\@ifundefined{C}%
{\newcommand{\C}{\ensuremath{{\mathbb C}}}}%
{\renewcommand{\C}{\ensuremath{{\mathbb C}}}}
\makeatother

\newcommand{\Q}{\ensuremath{{\mathbb Q}}}
\newcommand{\N}{\ensuremath{{\mathbb N}}}
\newcommand{\Z}{\ensuremath{{\mathbb Z}}}

% Reynolds, Womersley number, etc.
\newcommand{\Reynolds}{\mathrm{Re}}
\newcommand{\Womersley}{\mathrm{Wo}}
\newcommand{\Rayleigh}{\mathrm{Ra}}
\newcommand{\RayleighSqrt}{\mathrm{R}}
\newcommand{\Prandtl}{\mathrm{Pr}}
\newcommand{\Grashof}{\mathrm{Gr}}
\newcommand{\Mach}{\mathrm{Ma}}
\newcommand{\Froude}{\mathrm{Fr}}
\newcommand{\Peclet}{\mathrm{Pe}}
\newcommand{\Eckert}{\mathrm{Ec}}
\newcommand{\Brinkman}{\mathrm{Br}}


% Symetric and antisymetric tensors
\newcommand{\asym}[1]{\ensuremath{\Asym \left( #1 \right)}}
\newcommand{\sym}[1]{\ensuremath{\Sym \left( #1 \right)}}

% Energy, free energy, entropy, temperature
\newcommand{\ienergy}{\ensuremath{e}}
\newcommand{\fenergy}{\ensuremath{\psi}} % free energy
\newcommand{\entropy}{\ensuremath{\eta}} % entropy
\newcommand{\enthalpy}{\ensuremath{h}} % enthalpy
\newcommand{\temp}{\ensuremath{\theta}} % temperature
\newcommand{\temptoref}{\ensuremath{\vartheta}} % (temperature - reference temperature)/(reference temperature)


% Derivatives, partial derivatives, covariant derivatives 
\newcommand{\pd}[2]{\ensuremath{\frac{\partial {#1}}{\partial {#2}}}}
\newcommand{\ppd}[2]{\ensuremath{\frac{\partial^2 {#1}}{\partial {#2^2}}}}
\newcommand{\dd}[2]{\ensuremath{\frac{\diff {#1}}{\diff {#2}}}}
\newcommand{\cd}[2]{\ensuremath{\frac{\diff^* {#1}}{\diff {#2}}}}
\newcommand{\ddd}[2]{\ensuremath{\frac{\diff^2 {#1}}{\diff {#2}^2}}}

% Upper convected (Oldroyd) derivative
\newcommand{\fid}[1]{\ensuremath{\accentset{\triangledown}{#1}}}
\newcommand{\fidd}[1]{\ensuremath{\accentset{\triangledown \! \triangledown}{#1}}}

% Lower convected derivative
\newcommand{\lfid}[1]{\ensuremath{\accentset{\meddiamond}{#1}}}
\newcommand{\lfidd}[1]{\ensuremath{\accentset{\meddiamond \! \meddiamond}{#1}}}

% Jaumann derivative
\newcommand{\jfid}[1]{\ensuremath{\accentset{\vartriangle}{#1}}}
\newcommand{\jfidd}[1]{\ensuremath{\accentset{\vartriangle \! \vartriangle}{#1}}}

% Green--Naghdi derivative
\newcommand{\gfid}[1]{\ensuremath{\accentset{\medsquare}{#1}}}
\newcommand{\gfidd}[1]{\ensuremath{\accentset{\medsquare \! \medsquare}{#1}}}

% Truesdell derivative
\newcommand{\tfid}[1]{\ensuremath{\accentset{\medcircle}{#1}}}
\newcommand{\tfidd}[1]{\ensuremath{\accentset{\medcircle \! \medcircle}{#1}}}

\makeatletter
\@ifpackageloaded{tensor}% tensor is a package for a better typesetting of tensors
{
\newcommand{\codev}[2]{\ensuremath{{#1}|\indices{_{#2}}}}
}{%
\newcommand{\codev}[2]{\ensuremath{\left. {#1} \right|_{#2}}}
}
\makeatother

\makeatletter
\@ifpackageloaded{tensor}% tensor is a package for a better typesetting of tensors
{
\newcommand{\contradev}[2]{\ensuremath{{#1}|\indices{^{#2}}}}
}{%
\newcommand{\contradev}[2]{\ensuremath{\left. {#1} \right|^{#2}}}
}
\makeatother


% Bessel and Kelvin functions

\newcommand{\BesselI}[2]{\ensuremath{{\mathrm I}_{#1}\left(#2\right)}} 
\newcommand{\BesselK}[2]{\ensuremath{{\mathrm K}_{#1}\left(#2\right)}}
\newcommand{\BesselJ}[2]{\ensuremath{{\mathrm J}_{#1}\left(#2\right)}}
\newcommand{\BesselY}[2]{\ensuremath{{\mathrm Y}_{#1}\left(#2\right)}}

\newcommand\BesselRoot[2]{\ensuremath{{\rm j}_{#1,#2}}}

\newcommand{\KelvinBer}[2]{\ensuremath{{\mathrm{ber}}_{#1}\left(#2\right)}} 
\newcommand{\KelvinBei}[2]{\ensuremath{{\mathrm{bei}}_{#1}\left(#2\right)}}
\newcommand{\KelvinKer}[2]{\ensuremath{{\mathrm{ker}}_{#1}\left(#2\right)}}
\newcommand{\KelvinKei}[2]{\ensuremath{{\mathrm{kei}}_{#1}\left(#2\right)}}

% Chebyshev polynominals
\newcommand{\Chebyshevp}[3]{\ensuremath{{\mathrm T}_{#1}^{#2}\left(#3\right)}} 
\newcommand{\Chebyshev}[2]{\Chebyshevp{#1}{}{#2}} 

% norms
\newcommand{\norm}[2][]{\ensuremath{\left\|#2\right\|_{#1}}}
\newcommand{\absnorm}[1]{\ensuremath{\left|#1\right|}}

% volume
\makeatletter
\@ifundefined{volume}{%
\newcommand{\volume}[1][\Omega]{\ensuremath{#1}}}%
{%
\renewcommand{\volume}[1][\Omega]{\ensuremath{#1}}}
\makeatother

\newcommand{\svolume}[1][\Omega]{\ensuremath{\partial #1}}
\newcommand{\volumee}{\diff \mathrm{V}}
\newcommand{\surfacee}{\diff \vec{S}}
\newcommand{\surfacees}{\diff \mathrm{S}}
\newcommand{\linee}{\diff \vec{x}}

% volume and surface integral
\newcommand{\intvolume}[2][\volume]{\int_{#1} #2\; \volumee}
\newcommand{\intsvolume}[2][\svolume]{\int_{#1} #2\; \surfacee}


% products
\newcommand{\tensortensor}[2]{\ensuremath{#1 \otimes #2}}
\makeatletter
\@ifpackageloaded{MnSymbol} % : as binary operator needs MnSymbol package
{
\newcommand{\tensordot}[2]{\ensuremath{#1 \vdotdot #2}} 
}{%
\newcommand{\tensordot}[2]{\ensuremath{#1 : #2}} 
}
\makeatother
\newcommand{\vectordot}[2]{\ensuremath{#1 \bullet #2}}
\newcommand{\vectorcross}[2]{\ensuremath{#1 \times #2}}

% function spaces
\newcommand{\scont}[2][\Omega]{\ensuremath{{\mathcal C}^{#2} \left(#1 \right)}}
\newcommand{\sdist}[1][\Omega]{\ensuremath{{\mathcal D} \left(#1 \right)}}
\newcommand{\sdistd}[1][\Omega]{\ensuremath{{\mathcal D}^\prime \left(#1 \right)}}

\newcommand{\scdiv}[1][\Omega]{\ensuremath{{\mathcal V} \left(#1 \right)}}

\newcommand{\loc}{\mathrm{loc}}

\newcommand{\slebl}[2]{\ensuremath{L}^{#1}_{\loc} \left(#2 \right)}
\newcommand{\sleb}[2]{\ensuremath{L}^{#1} \left(#2 \right)}


\newcommand{\ssob}[3]{\ensuremath{W}^{#1, #2} \left(#3 \right)}

% dualities, scalar products
\newcommand{\fadual}[4]{\left\langle #1, #2\right\rangle_{#3, #4}}
\newcommand{\fascal}[4]{\left\langle #1, #2\right\rangle_{#3, #4}}

% tensorial function
\newcommand{\tensorf}[1]{{\mathfrak{#1}}}
